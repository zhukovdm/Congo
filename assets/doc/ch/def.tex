\section{Definitions}

\vspace{0.5em}

\underline{Congo}
is a chess variant popular in the Netherlands.

\vspace{0.5em}

\underline{Board}

\vspace{0.5em}

\underline{Rank}

\vspace{0.5em}

\underline{File}

\vspace{0.5em}

\underline{Square}

\vspace{0.5em}

\underline{Piece}
is any figure on the board used to play the game.

\vspace{0.5em}

\underline{Capture}
is a move by any piece that removes from the board the opponent's piece.

\vspace{0.5em}

\underline{River}
is the middle $4^{\text{th}}$ row of the board. A piece that ends its move in
the river must leave it next turn or \textbf{drown}.

\vspace{0.5em}

\underline{Castle}
is a $3 \times 3$ square at each side of the board, namely
\begin{equation*}
\{
    \text{C1}, \text{C2}, \text{C3},
    \text{D1}, \text{D2}, \text{D3},
    \text{E1}, \text{E2}, \text{E3}
\}
\text{ and }
\{
    \text{C5}, \text{C6}, \text{C7},
    \text{D5}, \text{D6}, \text{D7},
    \text{E5}, \text{E6}, \text{E7}
\}.
\end{equation*}
