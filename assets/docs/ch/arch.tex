\section{Application architecture}

The entire application is divided into several projects.

\subsection{Congo.Core}

\textsf{Congo.Core} provides implementations of the high-level abstractions,
such as Player, Board and all kinds of Pieces. There are two types of player
instances, controlled by a real player and controlled by an artificial agent.
Each player might enforce the board to advise using certain algorithm.
Artificial agent uses this move to proceed.

The player
ensures that transition move is valid. According to the rules, there will be 8
different types of pieces. Each piece has its own behavior under certain
circumstances.

\vspace{0.5em}

Algoritm.cs provides implementations of the various algorithms and
strategies, such as random choise, negamax, iterative deepening, and
many more. Once a player decides to get an advise, the immediate board is
passed to a certain algorithm, which returns a move. The current scope includes
\textsf{NegaMax} and \textsf{Iterative Deepening} algorithms with possible
extension in mind.

\subsection{Congo.Core.MSTest}

Supporting project used for \textsf{Congo.Core} unit testing.

\subsection{Congo.CLI}

\textsf{Congo.CLI} project will implement simple command line interface The
game will be controlled via issuing commands. The following commands are
supported: \textsf{a} (advice), \textsf{h} (help), \textsf{m} (move) and
\textsf{p} (play). The detailed description on each command is provided upon
request, enter \textsf{h h} to get more information.

\subsection{Congo.CLI.MSTest}

Supporting project used for \textsf{Congo.CLI} unit testing.

\subsection{Congo.Server}

Not yet implemented.

\subsection{Congo.WFA}

Not yet implemented.

\subsection{Congo.WPF}

Not yet implemented.

\subsection{Project dependencies}

\begin{center}
\begin{tabular}{ l | l }
Congo.Core        & -          \\
Congo.Core.MSTest & Congo.Core \\
Congo.CLI         & Congo.Core \\
Congo.CLI.MSTest  & Congo.CLI  \\
Congo.Server      & Congo.Core \\
Congo.WFA         & Congo.Core \\
Congo.WPF         & Congo.Core
\end{tabular}
\end{center}
