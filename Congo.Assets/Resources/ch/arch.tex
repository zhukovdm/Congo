\section{Application architecture}

The entire application is divided into several projects.

\subsection{Congo.Core}

\textsf{Congo.Core} provides implementations of all essential parts for
constructing a game, such as \textsf{CongoPiece}, \textsf{CongoPlayer},
\textsf{CongoBoard} and \textsf{CongoGame}.

\vspace{0.5em}

\textsf{CongoPiece} is an abstract class used for valid move generation for
each kind of piece in the game. Piece objects implement pattern
\textsf{Singleton} and don't hold any instance members. Piece calculates its
valid moves being provided with the color, board and current position. Each
type has its own specific procedure to calculate moves. Piece has
\textsf{Id} method, that returns \textsf{PieceId}. This data type is visible
only within Piece class and is used for faster piece recognition. Operations
on types, such as \textsf{typeof(...)}, are generally avoided. Piece also has
method \textsf{Code}, which is used for efficient piece encoding into the board
after a move is performed.

\vspace{0.5em}

\textsf{CongoPlayer} is an abstract class used for collecting valid move from
the current board. In the constructor, the type obtains \textsf{BitScan}
enumerator from the board, enumerate all squares containing pieces and call
piece calculation procedure for each position. Player also knows position of a
lion and its color. There are two types of player instances, controlled by a
real player (Hi) and controlled by an artificial agent (Ai). Each player may
enforce the board to advise using certain algorithm. Artificial agent picks
this move to proceed. \textsf{CongoPlayer} prescribes implementation of the
method \textsf{GetValidMove} to its descendants, the implementation shall
ensure getting valid move even from the user.

\vspace{0.5em}

\textsf{CongoBoard} is a class simulating the Congo board. Internally, the
instance of the board is represented by three pieces of information. First two
are \textsf{ulong} words \textsf{white-} and \textsf{blackOccupied} telling,
whether particular square is occupied and by which player. The third is an
immutable array of 7 \textsf{uint} words representing one rank of the board.
As each piece is encoded by $4$-bit value, only 28 bits of information are
necessary to encode the entire rank. Therefore, addition and removal pieces is
done by simple bitwise operations. Also, the board class contains precalculated
possible leaps for each type of piece and each position. Lambda functions are
heavily used in this calculation. As a consequence, generation of the valid
leaps is a simple iteration over all possible for a given position and piece.
Not all valid moves are leaps, some of the pieces slide and such moves shall
be generated each time.

\vspace{0.5em}

\textsf{CongoGame} is a container for a game snapshot. It contains references
to the current board and players. The most important instance method of this
class is \textsf{Transition(CongoMove move)}, given valid move it constructs
new game with new board and new players. All parts of the game are immutable,
once the game is created, nothing can be modified in it. Such property is a
great helper in concurrent calculations.

\vspace{0.5em}

Other classes are implementations of various algorithms, please see the
section "Algorithms".

\subsection{Congo.Core.MSTest}

Supporting project used for \textsf{Congo.Core} unit testing.

\subsection{Congo.CLI}

\textsf{Congo.CLI} project implements simple command line interface. Commands
are parsed, validated and executed by a corresponding delegates retrieved from 
predefined hash tables. Lambda functions are heavily used in verification.

\subsection{Congo.CLI.MSTest}

Supporting project used for \textsf{Congo.CLI} unit testing.

\subsection{Congo.Server}

Not yet implemented.

\subsection{Congo.WFA/WPF}

Not yet implemented.

\subsection{Project dependencies}

\begin{center}
\begin{tabular}{ l | l }
Congo.Core        & -          \\
Congo.Core.MSTest & Congo.Core \\
Congo.CLI         & Congo.Core \\
Congo.CLI.MSTest  & Congo.CLI  \\
Congo.Server      & Congo.Core \\
Congo.WFA         & Congo.Core \\
Congo.WPF         & Congo.Core
\end{tabular}
\end{center}
